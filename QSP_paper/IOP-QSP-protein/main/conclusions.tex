\section{Conclusion}

In this work, we introduced a framework for efficient quantum state preparation that addresses key challenges in encoding chemical data like proteins on quantum computers. The proposed Iterated Sparse Approximation (ISA) framework demonstrated several important advantages while balancing the competing demands of gate count optimization, classical computational efficiency, and hardware constraints. The ISA framework achieves approximate quantum state preparation using approximately 25\% fewer CX gates compared to exact methods like UCG, while maintaining compatibility with linear nearest neighbor architectures common in current quantum hardware. This reduction in gate count is crucial for minimizing decoherence effects in near-term quantum devices. The presented results also demonstrated that ISA can prepare quantum states orders of magnitude faster than variational approaches, making it particularly suitable for large-scale applications in protein data encoding. The framework successfully prepared both random uniform quantum states up to 14 qubits and protein-encoded quantum states from the UniProtKB database with high fidelity. The method demonstrated consistent performance across different state preparation tasks, suggesting robust applicability to various quantum computing applications. The limitations of our approach include higher gate counts compared to variational methods and potential challenges in achieving extremely high fidelities. However, these limitations are offset by the method's classical computational efficiency and hardware adaptability. Future work could explore alternative sparse approximation strategies or hybrid approaches incorporating limited variational optimization to further improve gate count efficiency while maintaining the framework's computational advantages.